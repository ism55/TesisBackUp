Este brazo manipulador cuenta con seis grados de libertad, siendo tres correspondientes al "brazo" y controlados con motores DC, y los grados restantes, correspondientes a la mano, con motores de aeromodelismo. Cuenta tambi�n con una pinza neum�tica como herramienta final de control.\\

Las acciones de control son realizadas con ayuda de un microcontrolador dsPIC, quien act�a como servidor recibiendo v�a puerto serial las �rdenes de control mediante la interfaz Robocom. Este programa ajusta las constantes correspondientes a cada controlador PID para los motores DC y las instrucciones de movimiento para la pinza y los motores de aeromodelismo. Tambi�n ajusta las variables de calibraci�n para el c�lculo �ptimo del valor a transmitir en la trama y el periodo de muestreo utilizado en la conversi�n anal�gica-digital.\\

\subsection{Software Robocom}

Es una interfaz gr�fica realizada en C++


\begin{table}[H]
% Table generated by Excel2LaTeX from sheet 'Hoja1'
\begin{tabular}{|c|c|c|}
	\hline
	\multicolumn{ 3}{|c|}{SERVIDOR} \\
	\hline
	1er byte & 2do y 3er byte & \multicolumn{ 1}{|c|}{Descripci�n} \\
	
	Comando (8bits) & Dato (16bits) & \multicolumn{ 1}{|c|}{} \\
	\hline
	A9 &       Dato & Valor del A/D de articulaci�n A \\
	\hline
	
	B9 &       Dato & Valor del A/D de articulaci�n A* \\
	\hline
	
	C9 &       Dato & Valor del A/D de articulaci�n B \\
	\hline
	
	D9 &       Dato & Valor del A/D de articulaci�n B* \\
	\hline
	
	E9 &       Dato & Valor del A/D de articulaci�n C \\
	\hline
	
	F9 &       Dato & Valor del A/D de articulaci�n C* \\
	\hline
\end{tabular}  
\caption{Comandos emitidos por el controlador (servidor) como respuesta al cliente.}
\end{table}

\begin{table}

% Table generated by Excel2LaTeX from sheet 'Hoja1'
\begin{tabular}{|c|c|c|}
	\hline
	\multicolumn{ 3}{|c|}{CLIENTE} \\
	\hline
	1er byte & 2do y 3er Byte & \multicolumn{ 1}{|c|}{Descripci�n} \\
	\hline
	
	Comando (8 bits) & Dato (16 bits) & \multicolumn{ 1}{|c|}{} \\
	
	\multicolumn{ 1}{|c|}{E0} &         00 & Enable articulacion A \\
	
	\multicolumn{ 1}{|c|}{} &         01 & Disable articulacion A \\
	
	\multicolumn{ 1}{|c|}{} &         02 & Enable articulacion B \\
	
	\multicolumn{ 1}{|c|}{} &         03 & Disable articulacion B \\
	
	\multicolumn{ 1}{|c|}{} &         04 & Enable pinza neum�tica \\
	
	\multicolumn{ 1}{|c|}{} &         05 & Disable pinza neum�tica \\
	
	\multicolumn{ 1}{|c|}{} &         06 & Enable articulacion C \\
	
	\multicolumn{ 1}{|c|}{} &         07 & Disable articulacion C \\
	
	\multicolumn{ 1}{|c|}{} &         09 & Solicitud de la posici�n del A/D de articulaci�n A \\
	
	\multicolumn{ 1}{|c|}{} &         10 & Solicitud de la posici�n del A/D de articulaci�n B \\
	
	\multicolumn{ 1}{|c|}{} &         11 & Solicitud de la posici�n del A/D de articulaci�n C \\
	\hline
	
	F0 &       Dato & Escribir Periodo de muestreo \\
	\hline
	F1 &       Dato & Escribir Referencia de control de PID A (PID\_A.controlReference) \\
	\hline
	F2 &       Dato & Escribir Referencia de control de PID B (PID\_B.controlReference) \\
	\hline
	F3 &       Dato & Escribir Referencia de control de PID C (PID\_C.controlReference) \\
	\hline
	F4 &       Dato & Escribir Referencia de motor de modelismo D (OC1RS) \\
	\hline
	F5 &       Dato & Escribir Referencia de motor de modelismo E (OC2RS) \\
	\hline
	F6 &       Dato & Escribir Referencia de motor de modelismo F (OC3RS) \\
	\hline
	F7 &       Dato & Escribir t�rmino proporcional del PID A \\
	\hline
	F8 &       Dato & Escribir t�rmino proporcional del PID B \\
	\hline
	F9 &       Dato & Escribir t�rmino proporcional del PID C \\
	\hline
	FA &       Dato & Escribir t�rmino integral del PID A \\
	\hline
	FB &       Dato & Escribir t�rmino integral del PID B \\
	\hline
	FC &       Dato & Escribir t�rmino integral del PID C \\
	\hline
	FD &       Dato & Escribir t�rmino derivativo del PID A \\
	\hline
	FE &       Dato & Escribir t�rmino derivativo del PID B \\
	\hline
	FF &       Dato & Escribir t�rmino derivativo del PID C \\
	\hline
\end{tabular}  
\caption{Comandos emitidos desde el software Robocom (cliente) al brazo.}
\end{table}

