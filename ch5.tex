Las especificaciones del dise�o del sistema de acceso remoto requieren el establecimiento de un servidor web para realizar una comunicaci�n inal�mbrica desde una p�gina web con acceso desde un cliente. En esta p�gina web se desea implementar los comandos de movimiento y configuraci�n del brazo robot en tiempo real o con baja latencia para as� eliminar completa o parcialmente la presencia de un computador de forma cableada.

Para compensar la ausencia del computador, este dispositivo debe conectarse al controlador del brazo manipulador v�a puerto serial para as� establecer la comunicaci�n entre el servidor y el cliente hacia el robot.

Por este motivo, se considera que las necesidades del proyecto requieren de uno o varios dispositivos que cuenten con UART, medios de conexi�n por radiofrecuencia, memoria flash adaptable a las necesidades del c�digo, y puertos de prop�sito general suficientes para implementar el prototipo. Dichas de otro modo, se resumen los requerimientos en la tabla \ref{TAB:requerimientos}.


\begin{table}[H]
	\caption{Estimaci�n de requerimientos para el dise�o.}
	\label{TAB:requerimientos}
\begin{center}
		\begin{tabular}{|c|c|}
	\hline 
	Comunicaci�n por radiofrecuencia & S� \\ 
	\hline 
	Memoria flash & Por el �rden de los MB \\ 
	\hline 
	GPIO & Al menos 10 \\ 
	\hline 
	Costo & Econ�mico \\ 
	\hline 
	UART & S� \\ 
	\hline 
	Capacidad de programaci�n en lenguaje C & S� \\ 
	\hline 
\end{tabular} 
\end{center}
	
\end{table}

En el mercado existe una gran cantidad de dispositivos que cubren estos requerimientos y muchos otros no contemplados. Por ello se analizar�n algunos de los m�s importantes.

Raspberry Pi 3

Esta tarjeta de desarrollo cuenta con microprocesador ARM Cortex

Cypress CYW54907

Contiene un microprocesador ARM Cortex R4

ESP 8266

ESP 32
