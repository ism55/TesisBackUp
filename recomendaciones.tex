Como recomendaci�n se propone mejorar este prototipo a�adiendo conectividad a internet, en el cu�l se puede hacer uso de recursos en la nube para implementar una aplicaci�n web mucho m�s eficiente.

Por otra parte, se pueden implementar otros m�todos de control al sistema controlador. Esto es utilizando este prototipo que ofrece comunicaci�n remota para adquirir informaci�n sobre el movimiento del brazo y recopilarlas para realizar control mediante algoritmos de redes neuronales o aprendizaje de m�quinas.

Tambi�n puede hacerse uso de software de realidad aumentada para generar los comandos de movimiento a trav�s de una aplicaci�n que monitorea v�a c�maras web el brazo robot y a su vez transmitir esta comunicaci�n al m�dulo prototipo.

Una mejora de este prototipo podr�a realizarse al implementar la comunicaci�n bidireccional entre el brazo robot y el cliente. Ya que con esto puede utilizarse la informaci�n para realizar mejores controladores.

En cuanto a la librer�a de jQuery, puede utilizarse el particionamiento de la memoria est�tica del microcontrolador para guardar en la memoria del mismo, im�genes con mejor resoluci�n y las librer�as de mayor inter�s, como lo es la jQuery.

Se propone implementar la soluci�n realizada en este trabajo a trav�s de otros dispositivos que posean conectividad a trav�s de Wi-Fi, para de esta forma comparar la eficiencia entre estos dispositivos para lograr un manejo remoto �ptimo. 
