Se logr� dise�ar un prototipo de sistema de acceso remoto al controlador del brazo manipulador MA2000 ubicado en el LCID.

El brazo robot puede manejarse a trav�s del software Robocom desarrollado en \cite{juan1} comunic�ndose a trav�s de un convertidor USB-serial conectado al puerto serial de un microcontrolador Microchip dsPIC30F3011. El prototipo realizado en este proyecto agreg� una mejora en el manejo del brazo robot, permitiendo que este pueda utilizarse sin el software, accediendo a una p�gina web sin usar el m�dulo convertidor USB ? Serial FTDI232.

La utilizaci�n del microcontrolador ESP8266 para establecer comunicaci�n v�a Wi-Fi a trav�s de un servidor alojado en el sistema embebido, demostr� ajustarse a las necesidades del proyecto. Este microcontrolador ayud� a establecer una comunicaci�n en tiempo real en la generaci�n de movimientos.
