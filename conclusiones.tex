Luego de un estudio sobre las t�cnicas y metodolog�as com�nmente utilizadas para el manejo de brazos manipuladores, se determinaron las acciones del sistema de control del brazo manipulador MA2000. Con esta informaci�n se seleccion� el dispositivo ESP8266 ya que se ajust� a los requerimientos del dise�o. Se seleccion� la comunicaci�n v�a Wi-Fi como modo de acceso remoto para el prototipo propuesto.

Se logr� dise�ar un prototipo de sistema de acceso remoto al controlador del brazo manipulador MA2000 ubicado en el LCID que permiti� el movimiento de este sin la presencia del software original ni conexiones al�mbricas diferentes al puerto serial del microcontrolador.

El brazo robot puede manejarse a trav�s del software Robocom desarrollado en \cite{juan1} comunic�ndose a trav�s de un convertidor USB-serial conectado al puerto serial de un microcontrolador Microchip dsPIC30F3011 sin ninguna alteraci�n. No obstante el prototipo realizado en este proyecto agreg� una mejora en el manejo del brazo robot, permitiendo que este pueda utilizarse sin el software, logrando acceder al sistema de movimiento mediante el uso de una p�gina web sin usar el m�dulo convertidor USB-Serial FTDI232.

La utilizaci�n del microcontrolador ESP8266 para establecer comunicaci�n v�a Wi-Fi a trav�s de un servidor alojado en el sistema embebido, demostr� ajustarse a las necesidades del proyecto, que no requer�a un hardware de gran nivel para alcanzar el objetivo. Este microcontrolador ayud� a establecer una comunicaci�n en tiempo real en la generaci�n de nuevas tramas de movimiento.


Se desarroll� un documento digital que sirvi� como Manual para la configuraci�n y utilizaci�n del m�dulo de acceso remoto y como recurso para la validaci�n del funcionamiento del sistema dise�ado. Este manual puede accederse en l�nea a trav�s de la direcci�n ism55.github.io/ManualTesis y en el se encuentran los videos de funcionamiento, soluci�n a problemas y la utilizaci�n de las versiones web y Android de la interfaz de usuario. 