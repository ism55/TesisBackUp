\section{Verificaci�n de funcionamiento}

Para verificar el funcionamiento del sistema, se realizaron las siguientes pruebas:\\

Desde la p�gina web (Robocom web):

\begin{enumerate}
	\item Conectividad y conexiones: Se muestra el �rden de encendido de los equipos que comprenden al sistema. (video 1). En el video 4 otra vista del procedimiento.
	
	\item Movimientos b�sicos en tiempo real: En esta prueba se realizan el movimiento de cada motor por separado. En el video 2 se tendr� la vista del brazo robot, en el video 3 se tendr� la vista desde la p�gina web. 
	
	\item Prueba de probador de trayectorias web: Desde el video 6 se observa c�mo utilizar el probador de trayectorias.
	
	\item Ajuste de PID: En el video 11 se tiene la vista desde la p�gina web en donde se configuran los par�metros del controlador PID y se realiza el movimiento de los motores nuevamente. En el video 9 se observa el movimiento del brazo robot luego de estos ajustes.
	
	\item LEDs indicadores: En el video 8, se observa el funcionamiento de los LEDs indicadores. Se tiene a un LED fijo que indica que el servidor se encuentra activo y el m�dulo encendido. Por otra parte, se tiene otro LED fijo por unos segundos antes de apagarse (Este realiza esta tarea cada vez que se cargue la p�gina web). Luego se puede ver el estado parpadeante que surge cuando el m�dulo est� recibiendo una comunicaci�n desde la p�gina web.
	

\end{enumerate}

Desde la aplicaci�n Android (Robocom Android):

\begin{enumerate}
	\item Movimientos b�sicos: En el video 10 puede verse el movimiento del brazo robot desde la aplicaci�n, en el video 12 se observan los comandos enviados desde la aplicaci�n. Adem�s, se usan los comandos wait para generar pausas entre cada instrucci�n.
	
	\item Rutinas programadas: En el video 15 puede verse una muestra de rutina programada y sin ajuste de los controladores PID. En el video 13 se tiene una muestra de la utilizaci�n de la funci�n 'cargar archivo'.
	
	\item Creaci�n de rutinas: En el video 14 se muestra la creaci�n de rutinas en archivos con extensi�n *.txt.
	
\end{enumerate}


Para otra manera de visualizar el funcionamiento, se recomienda visitar el enlace: ism55.github.io/ManualTesis. Si no posee internet, puede abrir el archivo 'index.html' incluido en la carpeta ManualTesis.
%Comunicaci�n:

%\begin{enumerate}
%	\item Comunicaci�n web
	
%	\item Comunicaci�n por puerto serial
%\end{enumerate}


\section{Manual}

Se implement� un manual digital basado en una p�gina web el cual se anexa en el CD que contiene este documento. All� podr�n verse los videos mencionados en la secci�n anterior. Adem�s, puede accederse a este manual en l�nea a traves de la direcci�n web: ism55.github.io/ManualTesis



