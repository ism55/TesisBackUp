%==============================================================================%
\section{Planteamiento del problema}
El brazo manipulador que se encuentra en el Laboratorio de Control (LCID) actualmente cuenta con un controlador y una interfaz instalados en un computador que sirve de administrador de las acciones de control del sistema. No obstante, el brazo manipulador es de uso reducido porque necesita ajustes en cuanto al controlador y porque las plataformas computacionales se est�n haciendo obsoletas; por lo que se quiere a�adir a esta implementaci�n otro sistema para realizar acciones de control de forma remota.

%==============================================================================%
\section{Justificaci�n}
Este equipo permitir� reforzar conocimientos en el �rea de sistemas autom�ticos de control. Siendo este manipulador un dise�o enfocado como equipo de laboratorio en la Escuela de Ingenier�a El�ctrica (EIE)\cite{juan1}, es siempre importante una propuesta para mejorar la utilidad del mismo utilizando nuevas tecnolog�as relacionadas con accesos remotos.



%==============================================================================%
\section{Objetivos}


\subsection{Objetivo general}
Dise�ar un prototipo de sistema de acceso remoto al controlador del brazo
manipulador MA2000.

\subsection{Objetivos espec�ficos}
\begin{itemize}
	\item Documentar las t�cnicas y metodolog�as de acceso remoto empleadas com�nmente en brazos manipuladores.
	\item Determinar las acciones del sistema de control del brazo manipulador.
	\item Seleccionar los dispositivos necesarios para el dise�o.
	\item Seleccionar el modo de acceso remoto.
	\item Dise�ar el sistema de acceso remoto.
	\item Validar la propuesta.
	\item Realizar un manual del prototipo.
\end{itemize}

%==============================================================================%
\section{Alcance y limitaciones}

El presente trabajo estar� solo enmarcado en el dise�o del prototipo. Por lo tanto, no contempla una aplicaci�n pr�ctica a un sistema industrial real. Cualquier implementaci�n f�sica o ajuste mec�nico para la validaci�n no est� contemplada en este trabajo.

%==============================================================================%
\section{An�lisis de factibilidad}

Para la realizaci�n de este trabajo se cuenta con una documentaci�n ampliada del sistema de control del brazo manipulador en cuesti�n. Adem�s, existe suficiente informaci�n con respecto al internet de las cosas y manipulaci�n remota de sistemas. 

Por otra parte, al tratarse del dise�o de un prototipo, no requiere necesariamente una implementaci�n, por lo cual este trabajo se considera factible ya que disminuye los riesgos econ�micos y el autor considera que cuenta con la informaci�n necesaria. 

De ser necesario, cualquier costo adicional ser� asumido por el autor. Por �ltimo, el tiempo que se propone se considera suficiente para la realizaci�n de este trabajo.
